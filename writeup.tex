\documentclass{article}
\usepackage{graphicx} % Required for inserting images
\setcounter{secnumdepth}{0}
\usepackage[margin=1in]{geometry}
\title{COMP551 Assignment 1: Analysis and Discussion}

\author{Thomas Lewis, Gabe Woloz, Vivek Motta}
\date{February 2026}

\begin{document}

\maketitle

\section{Abstract}
% Abstract (100–250 words) Summarize the task, methodology, and key findings.
\begin{abstract}
    
\end{abstract}

\section{Introduction}
% Introduction (5+ sentences) Describe the dataset, the modeling approach, and the goals of the project.


\section{Data Preprocessing and Exploration}
% Data Preprocessing and Exploration (5+ sentences) Explain cleaning, transformations, and visualization.
\subsection{Handling missing values}
There were no missing values
\subsection{Scaling and encoding choices}
Only season and weather were encoded, and nothing was scaled (consider adding that?).
\subsection{Effect on numerical stability}
Numerical stability should decrease with one-hot encoding since more columns/features are being considered with the risk of collinearity.

\section{Methods}
% Describe your linear regression implementation and feature engineering approach.
\subsection{Linear Regression Implementation}
\subsection{Transformations evaluated}
Polynomials, logs, interactions and sin cosine transformations were evaluated. (Explain why each and on what features)

\section{Results}
\begin{figure}[h]
    \centering
    \includegraphics[width=1\linewidth]{cols.png}
    \includegraphics[width=1\linewidth]{MSEData2.png}
    \caption{MSE data}
    \label{fig:placeholder}
\end{figure}
% Report quantitative results (MSE) and include figures or tables where appropriate.
\subsection{Effect of preprocessing on model performance}
Preprocessing led to better performance: \\
MSE reduced for non feature-engineered data: $818182.6491 \to 706621.5348$ for daily data; $20144.9541 \to 19823.6807$ for hourly data. \\
MSE reduced for feature-engineered data (less so): $619000.2709 \to 581049.2566$ for daily data; $16083.4239 \to 15958.1248$ for hourly data.
\subsection{Effect of feature engineering on model performance}

\subsection{Numerical stability of the model}

\section{Discussion and Conclusion}
% Interpret results, discuss limitations, and suggest next steps.
\subsection{Signs of overfitting or capacity change}
Overfitting was measured with the ratio between the training and testing MSEs. Sin cosine transformations either reduced overfitting or kept it stable while others only increased it. 
\subsection{Bias vs variance considerations}
\subsection{External/unobserved factors and data limits}
\subsection{Practical improvements and further experiments}

\section{Statement of Contributions}
% Briefly describe how all team members contributed (1-3 sentences).

\section{Statement on the Use of LLMs}
% How did you use LLMs if at all for the project (1-3 sentences).

\end{document}